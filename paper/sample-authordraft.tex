%! Author = ashutosh
%! Date = 4/20/22

\documentclass{article}
\usepackage[letterpaper,top=2cm,bottom=2cm,left=3cm,right=3cm,marginparwidth=1.75cm]{geometry}
\usepackage[english]{babel}
\usepackage{float}
\usepackage{amsmath}
\usepackage{graphicx}
\usepackage{caption}
\usepackage{subcaption}
\usepackage{hyperref}
\hypersetup{
    colorlinks=true,
    citecolor=green,
    linkcolor=blue,
    filecolor=cyan,
    urlcolor=magenta,
    }
\usepackage[utf8]{inputenc}
\usepackage{fancyhdr}
\usepackage{listings}



\def\at#1{{\color{red}\textbf{AT fix this: #1}}\xspace}

\title{Investigating Structure of Bias Manifold and Bias Evolution over years}


\author{Ashutosh Tiwari\\
	\normalsize ashutiwa@iu.edu
}

\pagestyle{fancy}
\fancyhf{}
\lhead{Presented as a project for  MGMT ACCESS USE BIG DATA(FALL 2022)}
\rfoot{Page \thepage}
\begin{document}
\maketitle



\section{Background}

Bias is an inseparable part of output models of considerable size. This bias is generally the result of data itself on which model is trained. This bias is tackled in general at three different stages in previous attempts. Some works (example ~\cite{DBLP:journals/corr/BolukbasiCZSK16a}) suggest that probably we should try to fix this bias after the model is trained and then debias the generated model embeddings. There is one more approach to learn models  that are fair (free from accumulated bias in dataset). This requires having two "unfair" models and then training a new model in presence of an adversercial setting~\cite{kenna_using_2021}. However we think that all these methods are bandaid at best, because of these reasons \begin{itemize}
    \item These limit usability and scope of a dataset and set of modeling techniques that can be applied to same. Because both of these rely on first training the model, these are methods are not generic enough both to be able to train any model on the dataset.
    \item These are very inefficient in terms of computing needs because first they need to train the model and then remove bias from embeddings/model.

\end{itemize}

Therefore a third method is to rather remove biases from the dataset itself ~\cite{ravfogel_null_2020}. In this project our aim is to figure out the structure of different kinds of biases in datasets and that will hopefull help us figure out different metrics and methods to remove those from the dataset.



\noindent \textbf{Keywords.} Graph Embeddings, Dataset Creation, Learning Fairness, Measuring Bias


\section{Introduction}

As part of this project, I will be training different simple models (Glove, Word2vec) models on different datasets and then will try to see how the structure of bias evolves overtime in each of those cases. To measure bias I will be using WEAT (Word Embedding Association Test) which was introduced in ~\cite{caliskan_semantics_2017}. This uses the evaluation dataset I found here in this \href{https://github.com/chadaeun/weat_replication/tree/master/weat}{github repository}. However because this dataset is pretty small, depending on time I will see I can find other evaluation datasets as well.

On top of this I will try to evaluate this general assumption in most of the previous works that in general bias manifolds is general. For this purpose I will try to train all these different models and see if I can separate gender bias using different methods used for linear separability. For this I will be using the dataset used by ~\cite{garg_word_2018}.

\at{See if we can complete this}
As part of this project, it was also intended to figure out most biased documents in a dataset. There are in general two different methods of doing this. One is Jack Knife method ~\cite{https://doi.org/10.48550/arxiv.1709.06183} and other is using bias gradients ~\cite{brunet_understanding_2019}. There are pros and cons of both of these methods. On one hand Jackknife takes a lot of time to run but is very general. On the other hand bias gradients for a set of documents is comparatively cheap to calculate but this method can only be used with a Glove model.
For jackknife method I will try using word2vec and for bias gradients I will try using Glove.

I implemented python version for ~\cite{brunet_understanding_2019} using this \href{https://github.com/mebrunet/understanding-bias}{official repository} to calculate bias gradient for a document given a glove model and cooccurance matrix of same. I hope that I will have time to use this to figure out most biased documents in datasets I use for this project. This will help us determine if bias learned by glove models can be verified by human annotators.




\section{Methodology}

I first divide the project into two parts. These are listed below:

\begin{itemize}
    \item \textbf{Part 1:} In this part I will be training different models on different datasets and then will try to see how extend of bias increases or decreases over time.
    \item \textbf{Part 2:} This part to is to verify or see if (gender) bias manifold itself is itself linear or not in nature.
    \item \textbf{Part 3:} In this part I take a datasets and then try to find most biased documents in that dataset using bias gradient calculation suggested in ~\cite{brunet_understanding_2019}.

\end{itemize}

Different models I will be using for this project are listed below:

\begin{itemize}
    \item Google pretrained word2vec 300 dimensional \href{
https://www.kaggle.com/datasets/leadbest/googlenewsvectorsnegative300}{model}
    \item Word2vec (implemented using Gensim)
    \item Glove (implemented taken from \href{https://github.com/stanfordnlp/GloVe}{here}, however wrapper on top of this C code is written in python)
\end{itemize}

Datasets I used for this project are listed below:
\begin{itemize}
    \item New York Times (NYT) dataset. Its a collection of 100 years of NYT articles and is available \href{https://www.kaggle.com/datasets/tumanovalexander/nyt-articles-data}{here}. We use this to see how bias manifold evolves over time after creating a merged dataset for every decade. That new dataset is available \href{https://www.kaggle.com/datasets/alphadraco/i535-new-dataset}{here}.
    \item Second is Wikipedia dataset. It is used to figure out the most biased document in the dataset.
\end{itemize}
All the code is avaliable as github repository \href{https://github.com/thunderock/bias_manifold}{here}. Trained model, along with code are available as this kaggle dataset \at{add link here}. All experiments are part of different notebooks which are present in notebooks folder of the github repository.

\subsection{Technological Setup}

\subsubsection{Programming Language and Code Structure}

As programming language I use python. Models folder contains three different models, glove, word2vec and a custom model, which can be used to load any pretrained model.

Glove is which is the official code is written in C. Everytime a Glove model is trained, the binaries are build and then used to generate vocab file, cooccurance matrix and model. Word2vec is implemented using gensim.

\subsubsection{Infrastructure}

Creation of aggregated dataset was done on Kaggle. Because training 10 different models for each decade was time taken, it was done on Carbonate.

Also to calculate bias gradient and score jackknife  of the documents I had to use Carbonate because it takes a lot of time to complete.

\subsubsection{Libraries and Packages}



\section{Results}

\section{Discussion}

\section{Conclusion}

\section{Future Work}


\begin{acks}
\end{acks}



\bibliographystyle{alpha}

\bibliography{sample-base}

\end{document}
\endinput